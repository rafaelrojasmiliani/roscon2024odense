
\section{Design}
\begin{frame}[t]
	\frametitle{Mathematical Foundations}

	We consider the problem of the minimization of a \emph{convex} combination of norms of the derivatives

	\begin{equation*}
		\min \int_{t_0}^{t_f} \alpha_1 \left\|\diff{\qv}{t}\right\|^2 + \alpha_2 \left\|\diffk{\qv}{t}{2}\right\|^2 + \cdots  + \alpha_1 \left\|\diffk{\qv}{t}{k} \right\|^2 \d t
	\end{equation*}

	And the corresponding Euler-Lagrange equations

	\begin{equation*}
		-\alpha_1 \diffk{\qv}{t}{2}+ \alpha_2 \diffk{\qv}{t}{4}- \cdots +  {(-1)}^k \alpha_k \diffk{\qv}{t}{2k}= 0 \ \ \text{a.e.}
	\end{equation*}

	At the $i$-interval we have
	\begin{equation*}
		\qv(t) = \xv_{i_1} \basis_1 + \cdots +  \xv_{i_{2k}} \basis_{2k}
	\end{equation*}
\end{frame}
\begin{frame}[t]
	\frametitle{Mathematical Foundations}


	The solution of the corresponding Euler-Lagrange equations

	\begin{equation*}
		-\alpha_1 \diffk{\qv}{t}{2}+ \alpha_2 \diffk{\qv}{t}{4}- \cdots +  {(-1)}^k \alpha_k \diffk{\qv}{t}{2k}= 0 \ \ \text{a.e.}
	\end{equation*}
	\begin{itemize}
		\item  Number of intervals
		\item Number of basis
		\item Coefficients
	\end{itemize}
\end{frame}

\begin{frame}[t]
	\frametitle{Mathematical Foundations: Features}

	We can prove that the solution of these ODEs respect the desired properites
		{\fontsize{10}{5}
			\begin{itemize}
				\item Analytical consistency:
				      \begin{equation*}
					      -\alpha_1 \diffk{}{t}{2} \left[\diff{\qv}{t}\right] +  \cdots +  {(-1)}^k \alpha_k \diffk{}{t}{2k} \left[\diff{\qv}{t}\right] = 0 \ \ \text{a.e.}
				      \end{equation*}
				\item Algebraic consistency I:\@ Linear Operations
				      \begin{eqnarray*}
					      -\alpha_1 \diffk{}{t}{2} \left[\qv_1 + \qv_2\right]  + \cdots +  {(-1)}^k \alpha_k \diffk{}{t}{2k} \left[\qv_1 + \qv_2\right] = 0 \ \ \text{a.e.}
				      \end{eqnarray*}
				\item Algebraic consistance II:\@ Linear Scaling
				      \only<1>{
					      \begin{eqnarray*}
						      -\alpha_1 \diffk{}{t}{2} \left[\qv(\sigma t)\right]  + \cdots +  {(-1)}^k \alpha_k \diffk{}{t}{2k} \left[\qv(\sigma t)\right] = 0 \ \ \text{a.e.}
					      \end{eqnarray*}
				      }
				      \only<2>{
					      \begin{eqnarray*}
						      -\warntext{\mathbf{\alpha}_1'} \diffk{}{t}{2} \left[\qv(\sigma t)\right]  + \cdots +  {(-1)}^k \alpha_k \diffk{}{t}{2k} \left[\qv(\sigma t)\right] = 0 \ \ \text{a.e.}
					      \end{eqnarray*}
				      }
			\end{itemize}
		}
\end{frame}
\begin{frame}[t]
	\frametitle{Mathematical Foundations: Linear Time-scaling invariance}
	If we desire a linear-time-scaling invariant cost function, we need to choose
	\begin{equation*}
		\min \int_{t_0}^{t_f} \alpha_1 T^2 \left\|\diff{\qv}{t}\right\|^2 + \alpha_2 T^4 \left\|\diffk{\qv}{t}{2}\right\|^2 + \cdots  + \alpha_k T^{2k}\left\|\diffk{\qv}{t}{k} \right\|^2 \d t
	\end{equation*}
\end{frame}

\begin{frame}[fragile]
	\frametitle{Class Design: Overview}

	\begin{columns}
		\begin{column}{0.5\textwidth}
			\begin{lstlisting}[language=c++]
class GSpline {
private:
    Eigen::VectorXd time_interval_lengths_;
    Eigen::VectorXd coefficients_;
    double initial_time_;
    std::unique_ptr<Basis> basis_;
public:
    GSpline linear_scaling_new_execution_time(double _T) const;
    GSpline operator+(const GSpline& that);
    GSpline operator*(double num);
    GSpline derivative(std::size_t deg);
    bool operator()(double time) const;
};
    \end{lstlisting}
		\end{column}
		\begin{column}{0.5\textwidth}
			\begin{lstlisting}[language=C++]
class Basis {
protected:
    Eigen::VectorXd parameters_float_;
    Eigen::VectorXi parameters_int_;
public:
    virtual void eval_on_window(
                    ..., Eigen::VectorXd&_buff) const = 0;
    virtual void eval_derivative_on_window(
                    ..., Eigen::VectorXd& _buff) const = 0;
    virtual void eval_derivative_wrt_tau_on_window(
                    ..., Eigen::VectorXd& _buff) const = 0;

    virtual std::unique_ptr<Basis> clone() const = 0;
};
    \end{lstlisting}
		\end{column}
	\end{columns}
	\begin{center}
		\begin{tikzpicture}[scale=0.9]
			\tikzumlset{font=\fontsize{5}{4}\selectfont}
			\umlemptyclass{GSpline}
			\umlemptyclass[x=7, y=0]{Basis}
			\umlunicompo [angle1=-90, angle2=-140 ] {GSpline}{Basis}
			\umlemptyclass[x=3, y=-2]{BasisLagrange}
			\umlemptyclass[x=6, y=-2]{BasisLegendre}
			\umlemptyclass[x=9, y=-2]{CustomBasis}
			\umlVHVinherit[] {CustomBasis}{Basis}
			\umlVHVinherit[] {BasisLagrange}{Basis}
			\umlVHVinherit[] {BasisLegendre}{Basis}
		\end{tikzpicture}
	\end{center}
\end{frame}

\begin{frame}[fragile]
	\begin{itemize}
		\item Polynomial Basis solution of
		      \begin{eqnarray}
			      \min \int_{t_0}^{t_f} \left\|\diffk{\qv}{t}{k}\right\|^2 \d t
		      \end{eqnarray}
		      \begin{itemize}
			      \item  \Verb|basis::BasisLegendre|
			      \item  \Verb|basis::BasisLagrange|
		      \end{itemize}
		\item Non Polynomial basis
		      \begin{eqnarray}
			      \min \int_{t_0}^{t_f} \alpha \left\|\diffk{\qv}{t}{}\right\|^2  + (\alpha - 1 )\left\|\diffk{\qv}{t}{3}\right\|^2 \d t
		      \end{eqnarray}
		      \begin{itemize}
			      \item  \Verb|basis::Basis0101|
		      \end{itemize}
	\end{itemize}
\end{frame}

\begin{frame}[fragile]
	\frametitle{Design: General operations}
	We also desire to work with trajectories that are not gsplines
	\begin{itemize}
		\item The norm of a derivative ${\left\| \ddot \qv \right\|}^2$
		\item Non linear parametrization $ \qv \circ s $, where $\qv, s\in \text{\Verb|GSpline|}$
	\end{itemize}
	For this reason we also provide interface to construct expressions
	\begin{itemize}
		\item Addition and scalar multiplication
		\item Concatenating functions
		\item Compisition
	\end{itemize}
	\begin{center}
		\begin{tikzpicture}[scale=0.9]
			\tikzumlset{font=\fontsize{5}{4}\selectfont}
			\umlemptyclass[x=0, y=0]{FunctionBase}
			\umlemptyclass[x=-3, y=-2]{CustomFunction}
			\umlemptyclass[x=3, y=-2]{FunctionExpression}
			\umlVHVinherit[] {FunctionExpression}{FunctionBase}
			\umlVHVinherit[] {CustomFunction}{FunctionBase}
			\umlHVHunicompo[anchor1=east,angle1=0, arm1=1cm, angle2=0,mult1=1, mult2=1..*] {FunctionExpression}{FunctionBase}
		\end{tikzpicture}
	\end{center}
\end{frame}
\begin{frame}[fragile]
	\frametitle{Design: General Overview}
	\begin{center}
		\begin{tikzpicture}
			\tikzumlset{font=\fontsize{5}{4}\selectfont}
			\umlemptyclass[x=0, y=0]{FunctionBase}
			\umlemptyclass[x=3, y=-2]{FunctionExpression}
			\umlVHVinherit[] {FunctionExpression}{FunctionBase}
			\umlHVHunicompo[anchor1=east,angle1=0, arm1=1cm, angle2=0,mult1=1, mult2=1..*] {FunctionExpression}{FunctionBase}
			\tikzumlset{font=\fontsize{5}{4}\selectfont}
			\umlemptyclass[y=-4]{GSpline}
			\umlemptyclass[x=7, y=-4]{Basis}
			\umlunicompo [angle1=-90, angle2=-140 ] {GSpline}{Basis}
			\umlemptyclass[x=3, y=-6]{BasisLagrange}
			\umlemptyclass[x=6, y=-6]{BasisLegendre}
			\umlemptyclass[x=9, y=-6]{CustomBasis}
			\umlVHVinherit[] {CustomBasis}{Basis}
			\umlVHVinherit[] {BasisLagrange}{Basis}
			\umlVHVinherit[] {BasisLegendre}{Basis}
			\umlVHVinherit[] {GSpline}{FunctionBase}
		\end{tikzpicture}
	\end{center}
\end{frame}

